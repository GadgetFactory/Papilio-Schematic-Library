\section{SPI}
\subsection{HDL sources and modules}
\subsubsection{HDL instantiation template}
\begin{lstlisting}[language=VHDL]
component zpuino_spi is
  port (
    wb_clk_i: in std_logic;
    wb_rst_i: in std_logic;
    wb_dat_o: out std_logic_vector(wordSize-1 downto 0);
    wb_dat_i: in std_logic_vector(wordSize-1 downto 0);
    wb_adr_i: in std_logic_vector(maxIObit downto minIObit);
    wb_we_i:  in std_logic;
    wb_cyc_i: in std_logic;
    wb_stb_i: in std_logic;
    wb_ack_o: out std_logic;
    wb_inta_o:out std_logic;

    mosi:     out std_logic;
    miso:     in std_logic;
    sck:      out std_logic;
    enabled:  out std_logic
  );
end component;
\end{lstlisting}

\subsubsection{Compliance}
The SPI module is wishbone compatible, in non-pipelined mode.

\subsubsection{Source files}

\subsection{Location}

\subsection{Registers}

\subsubsection{SPICTL}
SPI control register.

\begin{table}[H]
\begin{center}
\begin{tabularx}{14cm}{XXccccXc}
31 \hfill 10 & 9 \hfill 8 & 7 & 6 & 5 & 4 & 3 \hfill 1 & 0 \\

\hline
\multicolumn{1}{|c|}{\tiny Reserved} &
\multicolumn{1}{|c|}{\tiny SPITS \footnote{don't use}} &
\multicolumn{1}{|c|}{\tiny SPIBLOCK} &
\multicolumn{1}{|c|}{\tiny SPIEN}  &
\multicolumn{1}{|c|}{\tiny SPISRE}  &
\multicolumn{1}{|c|}{\tiny SPICPOL}  &
\multicolumn{1}{|c|}{\tiny SPICP}  &
\multicolumn{1}{|c|}{\tiny SPIREADY}  \\

\hline
\end{tabularx}
\caption{SPICTL register}
\end{center}
\end{table}


\begin{description}
\item{0 - SPIREADY [RW]} \hfill \\ SPI Ready bit. Reads as 1 when transfer has been completed or module is idle, 0 otherwise. If you use blocking operation
then you don't have to check this bit.
\item{[3:1] - SPICP [RW]}\hfill \\  SPI clock prescaler. The prescaler has 3 bits.
\item{4 - SPICPOL [RW]}\hfill \\  SPI clock polarity.
\item{5 - SPISRE [RW]}\hfill \\  SPI capture edge. If set to 1, then input is latched on rising clock edge, otherwise it's latched on falling edge.
\item{6 - SPIEN [RW]}\hfill \\  SPI enable bit. Must be set to 1 for SPI operation.
\item{7 - SPIBLOCK [RW]}\hfill \\  SPI blocking operation. If this bit is set to 1, all reads and writes to SPI module will cause CPU to wait if transmission is
still in progress. This allows for fast transfers because you don't have to check for SPIREADY bit.\footnote{Note: interrupts are disabled when CPU is waiting for transmission to complete. This might cause interrupt jitter.}
\item{[9:8] - SPITS [RW]}\hfill \\  SPI Transfer size. Will be deprecated in a next release.
\end{description}


\subsubsection{SPIDATA}
SPI data register.

\begin{table}[H]
\begin{center}
\begin{tabularx}{14cm}{X}
31 \hfill 0 \\

\hline
\multicolumn{1}{|c|}{SPIDATA} \\

\hline
\end{tabularx}
\caption{SPIDATA register}
\end{center}
\end{table}

\begin{description}
\item{[31:0] - SPIDATA [RW]} \hfill \\ SPIDATA holds the transmission values. A write to this register will cause SPI to transmit the relevant bits, according to transfer size value.
The internal receive register is shifted left by the transfer size bits.
\end{description}





\subsection{SPI clock prescaler}
When enabled, the SPI clock prescaler divides the main clock according to table \ref{spiprescaler}.

\begin{table}[H]
\begin{center}
\begin{tabularx}{6cm}{|c|c|}
\hline
Prescaler value & Clock divider \\
\hline
000 & 1 \\
\hline
001 & 2 \\
\hline
010 & 4 \\
\hline
011 & 8 \\
\hline  
100 & 16  \\
\hline
101 & 64  \\
\hline
110 & 256 \\
\hline
111 & 1024 \\
\hline
\end{tabularx}
\caption{SPI prescaler values}\label{spiprescaler}
\end{center}
\end{table}



\subsection{Software}
